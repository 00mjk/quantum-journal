\documentclass[a4paper,noarxiv,onecolumn]{quantumarticle}
\usepackage[utf8]{inputenc}
\usepackage[english]{babel}
\usepackage[T1]{fontenc}

\usepackage{lipsum}
\usepackage{mdframed}
\usepackage{booktabs}
\usepackage{longtable}

\newcommand{\todo}[1]{\textcolor{red}{TODO: #1}}

\newcommand*{\ditto}{--- \raisebox{-0.5ex}{''} ---}

\newenvironment{options}
	{\medskip\noindent\begin{longtable}{p{.20\columnwidth}p{.744\columnwidth}}
	%\toprule
	\textsf{Option} & \textsf{Description} \\
	\midrule
	}
	{\bottomrule\end{longtable}}

\newcommand{\option}[2]{
	\small\texttt{#1} & {\small#2} \\
}

\newcommand{\compatibilityoption}[2]{
	\small\texttt{\textcolor{quantumgray}{#1}} & \small\textcolor{quantumgray}{#2} \\
}

\newcommand{\defaultoption}[2]{
\small\texttt{\textcolor{blue}{#1}} & \small\textcolor{blue}{#2} \\
}

%\newenvironment{options}
%{\begin{mdframed}}
%	{\end{mdframed}}
%
%\newcommand{\option}[2]{
%	\noindent
%	\begin{minipage}[t]{.18\columnwidth}
%		\small\texttt{#1}
%	\end{minipage}
%	\begin{minipage}[t]{.82\columnwidth}
%		\small#2
%	\end{minipage}
%}

\begin{document}
	\title{Documentation of quantumarticle.cls}
	\maketitle
	
	\tableofcontents
	
	\section{Introduction}
	
	\section{Package options}
	
	\subsection{Document setup}
	
	\subsubsection{Document format}
	The document class does not specify a default document format, an error is raised if none is specified.
	
	\begin{options}
		\option{a4paper}{
			Sets the paper size to A4.
		}
		\option{a5paper}{
			Sets the paper size to A5.
		}
		\option{b5paper}{
			Sets the paper size to B5.
		}
		\option{letterpaper}{
			Sets the paper size to letter. \todo{official name}
		}
		\option{legalpaper}{
			Sets the paper size to legal. \todo{official name}
		}
		\option{executivepaper}{
			Sets the paper size to executive. \todo{official name}
		}
	\end{options}

	\subsubsection{Orientation}
	The default orientation is portrait. 
	
	\begin{options}
		\option{landscape}{
			Sets the orientation of the document to landscape.
		}
	\end{options}

	\subsubsection{Fontsize}
	The default fontsize is 10pt.
	
	\begin{options}
		\defaultoption{10pt}{
			Sets the normal fontsize to 10pt.
		}
		\option{11pt}{
			Sets the normal fontsize to 11pt.
		}
		\option{12pt}{
			Sets the normal fontsize to 12pt.
		}
	\end{options}

	\subsubsection{Sides}
	The default siding is oneside \todo{better word than siding}.
	
	\begin{options}
		\defaultoption{oneside}{
			Prepares the document as onesided.
		}
		\option{twoside}{
			Prepares the document as twosided.
		}
	\end{options}

	\subsubsection{Titlepage}
	By default, the document class is typset with no extra titlepage.
	
	\begin{options}
		\option{titlepage}{
			Displays the title on a separate titlepage.
		}
		\compatibilityoption{notitlepage}{
			Displays the title on the same page as the text.
		}
	\end{options}

	\subsubsection{Draft}
	By default, the document class is typset as final.
	
	\begin{options}
	\option{draft}{
		Marks the document as a draft.
	}
	\defaultoption{final}{
		Does not mark the document as a draft.
	}
	\end{options}

	\subsection{Style options}
	
	\subsubsection{Equations}
	By default, equations are typset centered, with equation numbers on the right. \todo{Clear fleqn behaviour} 
	
	\begin{options}
		\option{fleqn}{
			Display equations left aligned.
		}
		\option{leqno}{
			Display equation numbers to the left of the equation.
		}
	\end{options}

	\subsubsection{Bibliography}
	
	\begin{options}
		\option{openbib}{
			\todo{Find out what openbib does.}
		}
	\end{options}

	\subsection{Pre-loaded packages}
	The document class only loads packages necessary for its operation, such as \texttt{geometry} or \texttt{hyperref}. For compatibility reasons, there are also commands that load certain widely used classes for the user.
	
	\begin{options}
		\option{amsfonts}{
			Load the package \texttt{amsfonts}.
		}
		\compatibilityoption{noamsfonts}{
			Don't load the package \texttt{amsfonts}.
		}
		\option{amssymb}{
			Load the package \texttt{amssymb}.
		}
		\option{amssymbol}{
			\ditto
		}
		\compatibilityoption{noamssymb}{
			Don't load the package \texttt{amssymb}.
		}
		\option{amsmath}{
			Load the package \texttt{amsmath}.
		}
		\compatibilityoption{noamsmath}{
			Don't load the package \texttt{amsmath}.
		}
	\end{options}

	\subsection{Publication}
	By default, all documents that are created with the \texttt{quantumarticle} documentclass are treated as possible submissions to Quantum. 
	
	\begin{options}
		\option{accepted=YYYY-MM-DD}{
			Adds the note 'Accepted in Quantum YYYY-MM-DD' to the document.
		}
		\option{unpublished}{
			Intended for works that are not published in Quantum. Disables all Quantum-related branding.
		}
		\option{noarxiv}{
			Intended for works that are not meant to be uploaded to the arXiv. Disables all arXiv-related sanity checks in the document. This option also sets \texttt{unpublished}.
		}
	\end{options}

	\subsection{Compatibility}
	The \texttt{quantumarticle} class aims to be maximally compatible with documents that were previously typset with other document classes. For this reason, a lot of options are present for the sole purpose of compatibility but don't have an effect on how your document will be typset. These include:
	
	\begin{options}
		\compatibilityoption{checkin}{No effect.}
		\compatibilityoption{preprint}{\ditto}
		\compatibilityoption{reprint}{\ditto}
		\compatibilityoption{manuscript}{\ditto}
		\compatibilityoption{noshowpacs}{\ditto}
		\compatibilityoption{showpacs}{\ditto}
		\compatibilityoption{showkeys}{\ditto}
		\compatibilityoption{noshowkeys}{\ditto}
		\compatibilityoption{balancelastpage}{\ditto}
		\compatibilityoption{nobalancelastpage}{\ditto}
		\compatibilityoption{nopreprintnumbers}{\ditto}
		\compatibilityoption{preprintnumbers}{\ditto}
		\compatibilityoption{hyperref}{\ditto}
		\compatibilityoption{bibnotes}{\ditto}
		\compatibilityoption{nobibnotes}{\ditto}
		\compatibilityoption{footinbib}{\ditto}
		\compatibilityoption{nofootinbib}{\ditto}
		\compatibilityoption{altaffilletter}{\ditto}
		\compatibilityoption{altaffilsymbol}{\ditto}
		\compatibilityoption{superbib}{\ditto}
		\compatibilityoption{citeautoscript}{\ditto}
		\compatibilityoption{longbibliography}{\ditto}
		\compatibilityoption{nolongbibliography}{\ditto}
		\compatibilityoption{eprint}{\ditto}
		\compatibilityoption{noeprint}{\ditto}
		\compatibilityoption{author-year}{\ditto}
		\compatibilityoption{numerical}{\ditto}
		\compatibilityoption{galley}{\ditto}
		\compatibilityoption{raggedbottom}{\ditto}
		\compatibilityoption{tightenlines}{\ditto}
		\compatibilityoption{lengthcheck}{\ditto}
		\compatibilityoption{reprint}{\ditto}
		\compatibilityoption{eqsecnum}{\ditto}
		\compatibilityoption{secnumarabic}{\ditto}
		\compatibilityoption{floats}{\ditto}
		\compatibilityoption{stfloats}{\ditto}
		\compatibilityoption{endfloats}{\ditto}
		\compatibilityoption{endfloats*}{\ditto}
		\compatibilityoption{osa}{\ditto}
		\compatibilityoption{osameet}{\ditto}
		\compatibilityoption{opex}{\ditto}
		\compatibilityoption{tops}{\ditto}
		\compatibilityoption{josa}{\ditto}
		\compatibilityoption{byrevtex}{\ditto}
		\compatibilityoption{floatfix}{\ditto}
		\compatibilityoption{nofloatfix}{\ditto}
		\compatibilityoption{ltxgridinfo}{\ditto}
		\compatibilityoption{outputdebug}{\ditto}
		\compatibilityoption{raggedfooter}{\ditto}
		\compatibilityoption{noraggedfooter}{\ditto}
		\compatibilityoption{frontmatterverbose}{\ditto}
		\compatibilityoption{linenumbers}{\ditto}
		\compatibilityoption{nomerge}{\ditto}
		\compatibilityoption{hypertext}{\ditto}
		\compatibilityoption{frontmatterverbose}{\ditto}
		\compatibilityoption{inactive}{\ditto}
		\compatibilityoption{groupedaddress}{\ditto}
		\compatibilityoption{unsortedaddress}{\ditto}
		\compatibilityoption{runinaddress}{\ditto}
		\compatibilityoption{superscriptaddress}{\ditto}
		\compatibilityoption{aps}{Triggers revtex compatibility mode}
		\compatibilityoption{pra}{\ditto}
		\compatibilityoption{prb}{\ditto}
		\compatibilityoption{pre}{\ditto}
		\compatibilityoption{prl}{\ditto}
		\compatibilityoption{prx}{\ditto}
		\compatibilityoption{aip}{\ditto}
	\end{options}
		
	\section{Further information}
\end{document}